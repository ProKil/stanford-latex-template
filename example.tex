\pdfoutput=1
\documentclass[11pt, a4paper, logo]{stanford}

\usepackage[authoryear, sort&compress, round]{natbib}
\usepackage{microtype}
\usepackage{hyperref}
\usepackage{url}
\usepackage{booktabs}
\usepackage{graphicx}
\usepackage{amsmath}
\usepackage{lipsum}

\definecolor{accentcolor}{HTML}{175E54}
\hypersetup{
	colorlinks=true,
	citecolor=accentcolor,
	linkcolor=accentcolor,
	urlcolor=accentcolor
}

\title{Stanford LaTeX Template: Example Document}

\author{Your Name\\
Stanford University\\
\texttt{yourname@stanford.edu}}

\begin{abstract}
This is an example document demonstrating the Stanford LaTeX template. The template is based on a clean, professional academic paper format with Stanford branding.
\end{abstract}

\begin{document}
\maketitle

\tableofcontents

\section{Introduction}

\lipsum[1-2]

\subsection{Background}

\lipsum[3]

\section{Methods}

Here is an example equation:
\begin{equation}
E = mc^2
\end{equation}

\lipsum[4]

\section{Results}

\begin{table}[h]
\centering
\caption{Example results table}
\label{tab:example}
\begin{tabular}{lcc}
\toprule
Method & Accuracy & Time (s) \\
\midrule
Method A & 0.95 & 10.2 \\
Method B & 0.87 & 5.3 \\
Method C & 0.92 & 7.8 \\
\bottomrule
\end{tabular}
\end{table}

\lipsum[6]

\section{Discussion}

\lipsum[7-8]

\section{Conclusion}

\lipsum[9]

\section*{Acknowledgments}

This template is adapted from the Google DeepMind LaTeX template and customized for Stanford University documents.

\end{document}
